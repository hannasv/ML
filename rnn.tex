

\begin{figure}[hp]
    \centering
        \begin{tikzpicture}[
    % GLOBAL CFG
    font=\sf \scriptsize,
    >=LaTeX,
    % Styles
    cell/.style={% For the main box
        rectangle, 
        rounded corners=5mm, 
        draw,
        very thick,
        },
    operator/.style={%For operators like +  and  x
        circle,
        draw,
        inner sep=-0.5pt,
        minimum height =.01cm,
        },
    function/.style={%For functions
        ellipse,
        draw,
        inner sep=1pt
        },
    ct/.style={% For external inputs and outputs
        circle,
        draw,
        line width = .75pt,
        minimum width=0.2cm,
        inner sep=1pt,
        },
    gt/.style={% For internal inputs
        rectangle,
        draw,
        minimum width=4mm,
        minimum height=3mm,
        inner sep=1pt
        },
    mylabel/.style={% something new that I have learned
        font=\scriptsize\sffamily
        },
    ArrowC1/.style={% Arrows with rounded corners
        rounded corners=.25cm,
        thick,
        },
    ArrowC2/.style={% Arrows with big rounded corners
        rounded corners=.5cm,
        thick,
        },
    ]

%Start drawing the thing...    
    % Draw the cell: 
    \node [cell, minimum height =1.5cm, minimum width=2cm, fill = cyan!50] (first) at (0,0){\Large \textbf{A}}; % , fill=green

    %\node[ct, label={[mylabel]Cell state}] (c) at (-4,1.5) {\empt{c}{t-1}};
    \node[ct, fill = red!50, scale = 2.25] (h) at (0, 2) {$y_{t}$}; % , fill=blue
    \node[ct, fill = green!50, scale = 2.25] (x) at (0, -2) { $x_t$}; %, fill = magenta
    \draw [->, ArrowC1] (x) -- (first);
    \draw [->, ArrowC1] (first) -- (h);

    %\node [operator, fill = black, opacity = 1] (a) at (0, 1) { };
    %\node [operator, fill = black, opacity = 1] (d) at (-2, 1) { };
    %\node [operator, fill = black, opacity = 1] (c) at (2, 0) { }; 
    %\node [operator, fill = black, opacity = 1] (b) at (-2, .0) { };

    %\draw [->, ArrowC1] (first) -- (a) -- (d) -- (b)  -- (first);
     
    \draw[-latex, thick, white] (first) to[out=110,in=180, loop] node[above] { \Large $h_t$} (first);  %
    \draw[-latex, thick, black] (first) to[out=70,in=360, loop] node[auto] {\Large $h_t$} (first);  

    %\draw [->] (first) to[loop above] node[auto] {} (first);
    \end{tikzpicture}
    
    \caption{Simple one layer recurrent network. $x_t$ denotes the input, $h_t$ denotes the output/ hidden state of the neuron and A denotes a artificial recurrent unit. Note the output can be different from the hidden state, but it doesn't necessarily have to be. Inspired by \href{http://colah.github.io/posts/2015-08-Understanding-LSTMs/}{http://colah.github.io/posts/2015-08-Understanding-LSTMs/}.}
    \label{fig:rnn}
\end{figure}